\documentclass[12pt]{article}
\usepackage[top=2cm]{geometry}
\usepackage{amsmath}
\usepackage{amsthm}
\usepackage{amsfonts}
\usepackage{parskip}
\usepackage{graphicx}


\newtheorem{definition}{Definizione}[section]
\newtheorem{proposition}{Proposizione}[section]
\newtheorem{theorem}{Teorema}[section]
\newtheorem{nota}{Nota}[section]
\newtheorem{notaAMargine}{Nota a margine}[section]

\title{\textbf{Modelli di variabili aleatorie}}
\author{Daniele Falanga}
\date{}

\begin{document}
\maketitle

\section{Variabili aleatorie di Bernoulli e binomiali}
Le variabili aleatorie di Bernoulli e binomiali caratterizzano tutta quella classe
di esperimenti il cui esito può essere solo "successo" o "fallimento"

\begin{definition}[Bernoulli]
    Una variabile aleatoria X si dice di Bernoulli se la sua funzione di massa
    di probabilità è del tipo: 
    \begin{align*}
        P(X = 0) = & 1-p \\
        P(X=1) = &  p \\
    \end{align*}

    La variabile X di Bernoulli può assumere solo valori 0 e 1. 
    \newline
    Il suo valore atteso è:
    \[
    E[X] = p    
    \]
\end{definition}

Se invece si ripete l'esperimento \(n\) volte, indipendentemente dal risultato, 
Si usa la variabile binomiale e deve rispettare i seguenti requisiti:
\begin{itemize}
    \item Il risultato dell' evento può essere solo positivo o negativo
    \item ciascun evento è indipendente dagli altri
    \item Il processo/variabile può assumere un determinato e fissato numero di valori
    \item La probabilità di di successo o fallimento di un evento è costante
\end{itemize}

\begin{definition}[binomiale]
    La definizione alla base è analoga a quella di Bernoulli, con l'aggiunta dei parametri
    \((n,p)\). 
    \newline
    La funzione di massa di probabilità è: 
    \begin{align*}
        P(X = i) = \binom{n}{i}p^i(1-p)^{n-i}, & & i = 0,1\dots,n
    \end{align*}
\end{definition}

Ogni successione con \(i\) successi e \(n-i\) insuccessi ha probabilità \(p^i(1-p)^{n-i}\), mentre il numero
di queste successioni, pari al numero di combinazioni in cui può esssere svolto l'esperimento, 
è dato dal cofficente binomiale.

\newpage
Il valore atteso di una binomiale:
\[
E[X] = np    
\]

La Varianza:
\[
Var(X) = np(1-p)    
\]

\section{Variabili aleatorie di Poisson}
La distribuzione di Poisson è una distribuzione discreta che esprime la probabilità
che per il numero di eventi che si verificano successivamente ed indipendentemente in un 
dato intervallo di tempo, sapendo che mediamente se ne verificano un numero \(\lambda\) a volte 
definito negli esercizi come il valore atteso.

\begin{definition}
    La distribuzione di Poisson è data da:
    \begin{align*}
        P(X=i) = \frac{\lambda^i}{i!}e^{-\lambda}, && \quad i=0,1,2,\dots
    \end{align*}
\end{definition}



\end{document}