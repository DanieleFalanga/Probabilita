\documentclass[12pt]{article}
\usepackage[top=2cm]{geometry}
\usepackage{amsmath}
\usepackage{amsthm}
\usepackage{amsfonts}
\usepackage{parskip}
\usepackage{graphicx}


\newtheorem{definition}{Definizione}[section]
\newtheorem{proposition}{Proposizione}[section]
\newtheorem{theorem}{Teorema}[section]
\newtheorem{nota}{Nota}[section]
\newtheorem{notaAMargine}{Nota a margine}[section]

\title{\textbf{Modelli di variabili aleatorie}}
\author{Daniele Falanga}
\date{}

\begin{document}
\maketitle

\section{Variabili aleatorie di Bernoulli e binomiali}
Le variabili aleatorie di Bernoulli e binomiali caratterizzano tutta quella classe
di esperimenti il cui esito può essere solo "successo" o "fallimento"

\begin{definition}[Bernoulli]
    Una variabile aleatoria X si dice di Bernoulli se la sua funzione di massa
    di probabilità è del tipo: 
    \begin{align*}
        P(X = 0) = & 1-p \\
        P(X=1) = &  p \\
    \end{align*}

    La variabile X di Bernoulli può assumere solo valori 0 e 1. 
    \newline
    Il suo valore atteso è:
    \[
    E[X] = p    
    \]
\end{definition}

Se invece si ripete l'esperimento \(n\) volte, indipendentemente dal risultato, 
Si usa la variabile binomiale.

\begin{definition}[binomiale]
    La definizione alla base è analoga a quella di Bernoulli, con l'aggiunta dei parametri
    \((n,p)\). 
    \newline
    La funzione di massa di probabilità è: 
    \begin{align*}
        P(X = i) = \binom{n}{i}p^i(1-p)^{n-i}, & & i = 0,1\dots,n
    \end{align*}
\end{definition}



\end{document}