\documentclass[12pt]{article}
\usepackage[top=2cm]{geometry}
\usepackage{amsmath}
\usepackage{amsthm}
\usepackage{amsfonts}
\usepackage{parskip}
\usepackage{graphicx}


\newtheorem{definition}{Definizione}[section]
\newtheorem{proposition}{Proposizione}[section]
\newtheorem{theorem}{Teorema}[section]
\newtheorem{nota}{Nota}[section]
\newtheorem{notaAMargine}{Nota a margine}[section]

\title{\textbf{Foglio 8}}
\author{Daniele Falanga}
\date{}

\begin{document}
\maketitle

\subsection*{Esercizio 1}

\begin{enumerate}
    \item Per il primo punto:
    \[
    P(C) = \underbrace{P(AC)}_{\text{Collegamento diretto}} + \underbrace{P(AB)P(BC)}_{\text{Andando a B e poi a C da B}}    
    \]
    Essendo i collegamenti per C, passando per B, indipendenti l'un l'altro, si molitiplicano le probabilità
    \item Per il secondo, se si elimina la possibilitò di andare a C, per arrivare a B ci sta una possibilità di \(\frac{1}{2}\)
\end{enumerate}

\newpage
\subsection*{Esercizio 2}
Il calcolo della distribuzione segue questo schema, dette \(n,u,d\), il numero di pick di batterie, nuove, usate e difettose:
\[
P(X=n,Y=u) = \frac{\binom{3}{n} \binom{2}{u} \binom{2}{d}}{\binom{7}{3}}    \quad \text{Dove: } n+u+d = 3
\]
La distribuzione congiunta e le relative marginali:
\begin{equation*}
    \renewcommand{\arraystretch}{1.5}
    \begin{array}{c|cccc|c}
          & 0 & 1 & 2 & 3 & m\\
    \hline
    0 & 0 & \frac{2}{35} & \frac{2}{35} & 0 & \frac{4}{35}\\
    1 & \frac{3}{35} & \frac{12}{35} & \frac{3}{35} & 0 & \frac{18}{35} \\
    2 & \frac{6}{35} & \frac{6}{35} & 0 & 0 & \frac{12}{35}\\ 
    3 & \frac{1}{35} & 0 & 0 & 0 & \frac{1}{35}\\
    \hline
    m & \frac{10}{35} & \frac{20}{35} & \frac{5}{35} & 0 & 1 
    \end{array}
\end{equation*}

Utilizzando la relazione dell'indipendenza di variabili aleatorie:
\[
P(X=x,Y=y) = P(X=x)\cdot P(Y=y)    
\]
Noto che non è rispettata e quindi, le variabili sono dipendenti.
La varianza:
\[
Covar(X,Y) = E[XY] - E[X]E[Y] = -0.17    
\] 

Per il punto 3 faccio la somma delle probabilità:
\[
P(F) = P(3,0) + P(2,1) + P(1,2) = \frac{10}{35}    
\]
\subsection*{Esercizio 3}
\begin{align*}
    \begin{cases}
        &X = \text{num. comp sottoposti al controllo} \\
        &Y = \text{num. comp non funzionanti} \\
        &Z = \text{num. comp scartati dopo il controllo} \\
    \end{cases}
\end{align*}

Entrambe le 3 variabili possono essere modellate come una distribuzione binomiale per cui:
\begin{align*}
    &X = bin(n,\alpha) \\
    &X = bin(n,p) \\
    &X = bin(n,?) \\
\end{align*}

Avendo tutte una distribuzione binomiale:
\begin{align*}
    &X_i \rightarrow
    \begin{cases}
        1 \quad \text{se componente i-esimo è sottoposto al controllo} \\
        0 \quad \text{altrimenti}
    \end{cases}
\end{align*}

\begin{align*}
    &Y_i \rightarrow
    \begin{cases}
        1 \quad \text{se componente i-esimo è è difettoso} \\
        0 \quad \text{altrimenti}
    \end{cases}
\end{align*}

\begin{align*}
    &Z_i \rightarrow
    \begin{cases}
        1 \quad \text{se componente i-esimo è scartato dopo il controllo} \\
        0 \quad \text{altrimenti}
    \end{cases}
\end{align*}

Dobbiamo calcolare ora:
\begin{align*}
    &P(Z_i = 1) = P(Z_i = 1, Y_i = 1) = \\
    &P(Z_i = 1, Y_i = 1 | X_i = 1) + \underbrace{P(Z_i = 1, Y_i = 1 | X_i=0)}_{0} = \\
    &P(Z_i = 1, X_i = 1 | Y_i = 1)\cdot P(X_i = 1) = \\
    &P(Z_i = 1 | X_i = 1 \cap Y_i = 1)\cdot P(Y_i = 1) \cdot P(X_i = 1 | Y_i = 1) = 1\cdot p \cdot \alpha
\end{align*}

Avendo calcolato la probabilità che un componente singolo sugli n complessivi sia scartato posso affermare che:
\[
Z = bin(n,p\alpha)     
\]
\[
P(Z = k) = \binom{n}{k}\cdot (\alpha p)^k \cdot (1-\alpha p)^{n-k} \quad k = 0,1,\dots, n   
\]

{\bf{Il punto 2:}}  \newline
K scartati, n-k sono i componenti messi in commercio, e di questi, i sono i difettosi.
La variabile aleatoria V indica i componenti difettosi tra gli n-k componenti messi in commercio:
\[
P(V = i) = \binom{n-k}{i} p^i \cdot (1-p)^{n-k-i}    
\]
\subsection*{Esercizio 4}

La distribuzione segue una multinomiale:

\begin{equation*}
    \renewcommand{\arraystretch}{1.5}
    \begin{array}{c|ccc|c}
          & 0 & 1 & 2 & Y\\
    \hline
    0 & \frac{1}{36} & \frac{6}{36} & \frac{9}{36} & \frac{16}{36}\\
    1 & \frac{4}{36} & \frac{12}{36} & 0 & \frac{16}{36} \\
    2 & \frac{4}{36} & 0 & 0  & \frac{4}{36}\\ 
    \hline
    X & \frac{9}{36} & \frac{18}{36} & \frac{9}{36} & 1 
    \end{array}
\end{equation*}

La distribuzione di Z:
\begin{align*}
    &P(Z=0) = P(R=0,V=0) = \frac{1}{36} \\
    &P(Z=1) = P(R=0,V=1) + P(R=1,V=0) + P(R=1,V=1) = \frac{22}{36} \\
    &P(Z=2) = P(R=0,V=2) + P(R=2,V=0)  = \frac{13}{36} \\
\end{align*}
Il calcolo della varianza:
\begin{align*}
    &E[Z] = \frac{4}{3} \\
    &E[Z^2] = \frac{74}{36} \\
    &Var(Z) = \frac{5}{18}    
\end{align*}
    
\subsection*{Esercizio 5}
Calcolo le distribuzioni marginali:
\begin{align*}
    &P(Z=0) = P(X=0,Y,0)+P(X=0,Y,1)+P(X=1,Y,1) = p^2+(1-p)^2+p(1-p) \\
    &P(Z = 1) = P(X=1,Y,0) = p(1-p) \\
    &P(W = 0) = P(X=1,Y,1) = p^2 \\
    &P(W = 1) = P(X=0,Y,0)+P(X=0,Y,1)+P(X=1,Y,0) = (1-p)^2+2p(1-p) \\
\end{align*}

Le congiunte:
\begin{align*}
    &P(X = 0, Y = 0) = p^2 \\
    &P(X = 1, Y = 0) = 0 \\
    &P(X = 0, Y = 1) = (1-p)^2+(1-p)p \\
    &P(X = 1, Y = 1) = p(1-p) \\
\end{align*}

Eseguendo il prodotto delle marginali ed uguagliando il prodotto alle rispettive congiunte, si nota che 
le variabili sono indipendenti quando \(p = 0\)
\newpage
\subsection*{Esercizio 6}
B dichiara 1 con probabilità p, e 2 con probabilità 1-p. 
\begin{align*}
    X =
    \begin{cases}
        &1 \quad \text{se B indovina la scelta i di A} \\
        &0 altrimenti    
    \end{cases}
\end{align*}

B- scelta i-esima di B con:
\begin{align*}
    \begin{cases}
        &P(B = 1) = p \\
        &P(B = 2) = 1-p
    \end{cases}
\end{align*}

A- scrittura i-esimo numero su foglio \(i = 1,2\)

\begin{enumerate}
    \item Sapendo che \(A = 1\) (A ha scritto 1), si hanno due scenari possibili:
    il primo, B indovina che A ha scritto 1 e quindi guadagna 1 euro. \newline
    Il secondo che B non indovina e perde 0.75. \newline
    Il guadagno medio:
    \begin{align*}
        E[B_1] = 1P(B=1|A=1)-0.75P(B=2|A=1) = 1p-0.75(1-p) = 1.75p-0.75 \\
    \end{align*}
    \item Sapendo che \(A = 2\) (A ha scritto 2).
    Primo caso: B indovina e guadagna un euro
    Secondo caso: B sbaglia e perde 0.75
    \begin{align*}
        E[B_2] = 1P(B=2|A=2)-0.75P(B=1|A=2) = 2(1-p)-0.75p = 2-2.75p \\
    \end{align*}
    \item Risolvendo la seguente disequazione:
    \begin{align*}
        &1.75p-0.75 \ge 2-2.75p =\\
        &p \ge \frac{2.75}{4.50}
    \end{align*}
    Quindi si ha che la prima disequazione è maggiore della seconda se \(p \ge 0.61\) e viceversa per la seconda. 
    Quindi, la seconda disequazione viene massimizzata se \(p = 0.61\) mentre nel secondo caso, la prima disequazione è piu piccola,
    e viene massimizzata sempre quando \(p = 0.61\).
    \item La perdita media nel primo caso:
    \begin{align*}
        &E[A_1] = 0.75P(A_2|B_1) -1P(A_1|B_1) = 0.75-1.75q \\
    \end{align*}
    \item La perdita media nel secondo caso:
    \begin{align*}
        &E[A_2]= 0.75P(A_1|B_2) -2P(A_2|B_2) = 2.75q-2
    \end{align*}

    Risolvendo in modo analogo a prima trovo che la q cercata è proprio uguale a \(q = 0.61\)
    cioé lo stesso valore che massimizza il minimo dei guadagni di B, si può affermare che in tutti
    e due i casi il gioco è equilibrato. 
\end{enumerate}

\end{document}