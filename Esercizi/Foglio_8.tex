\documentclass[12pt]{article}
\usepackage[top=2cm]{geometry}
\usepackage{amsmath}
\usepackage{amsthm}
\usepackage{amsfonts}
\usepackage{parskip}
\usepackage{graphicx}


\newtheorem{definition}{Definizione}[section]
\newtheorem{proposition}{Proposizione}[section]
\newtheorem{theorem}{Teorema}[section]
\newtheorem{nota}{Nota}[section]
\newtheorem{notaAMargine}{Nota a margine}[section]

\title{\textbf{Foglio 7}}
\author{Daniele Falanga}
\date{}

\begin{document}
\maketitle

\subsection*{Esercizio 1}

\begin{enumerate}
    \item Per il primo punto:
    \[
    P(C) = \underbrace{\frac{1}{2}}_{\text{Collegamento diretto}} + \underbrace{\frac{1}{2} \cdot \frac{1}{2}}_{\text{Andando a B e poi a C da B}}    
    \]
    Essendo i collegamenti per C, passando per B, indipendenti l'un l'altro, si molitiplicano le probabilità
    \item PEr il secondo, se si elimina la possibilitò di andare a C, per arrivare a B ci sta una possibilità di \(\frac{1}{2}\)
\end{enumerate}

\newpage
\subsection*{Esercizio 2}
Il calcolo della distribuzione segue questo schema, dette \(n,u,d\), il numero di pick di batterie, nuove, usate e difettose:
\[
P(n,u) = \frac{\binom{3}{n} \binom{2}{u} \binom{2}{d}}{\binom{7}{3}}    
\]
La distribuzione congiunta e le relative marginali:
\begin{equation*}
    \renewcommand{\arraystretch}{1.5}
    \begin{array}{c|cccc|c}
          & 0 & 1 & 2 & 3 & m\\
    \hline
    0 & 0 & \frac{2}{35} & \frac{2}{35} & 0 & \frac{4}{35}\\
    1 & \frac{3}{35} & \frac{12}{35} & \frac{3}{35} & 0 & \frac{18}{35} \\
    2 & \frac{6}{35} & \frac{6}{35} & 0 & 0 & \frac{12}{35}\\ 
    3 & \frac{1}{35} & 0 & 0 & 0 & \frac{1}{35}\\
    \hline
    m & \frac{10}{35} & \frac{20}{35} & \frac{5}{35} & 0 & 1 
    \end{array}
\end{equation*}

Utilizzando la relazione dell'indipendenza di variabili aleatorie:
\[
P(X=x,Y=y) = P(X=x)\cdot P(Y=y)    
\]
Noto che non è rispettata e quindi, le variabili sono dipendenti.
La varianza:
\[
Covar(X,Y) = E[XY] - E[X]E[Y] = -0.17    
\] 

Per il punto 3 faccio la somma delle probabilità:
\[
P(F) = P(3,0) + P(2,1) + P(1,2) = \frac{10}{35}    
\]
\subsection*{Esercizio 3}

\begin{itemize}
    \item A = prodotto buono
    \item B = prodotto non funzionante
    \item E = controllato
    \item F = non controllato
\end{itemize}

\begin{enumerate}
    \item Primo punto:
    \begin{align*}
        &P(B|E) = \frac{P(E)P(E|B)}{P(B)} = \frac{P(E)P(E)}{P(B)} = \frac{\alpha^2}{p}
    \end{align*}
    \item Secondo punto:
    \begin{align*}
        &P(B|F) = \frac{P(F)P(F)}{P(B)} = \frac{(1-\alpha)^2}{p}
    \end{align*}
\end{enumerate}
\subsection*{Esercizio 4}

La distribuzione segue una multinomiale:

\begin{equation*}
    \renewcommand{\arraystretch}{1.5}
    \begin{array}{c|ccc|c}
          & 0 & 1 & 2 & m\\
    \hline
    0 & \frac{1}{36} & \frac{6}{36} & \frac{9}{36} & \frac{16}{36}\\
    1 & \frac{4}{36} & \frac{12}{36} & 0 & \frac{16}{36} \\
    2 & \frac{4}{36} & 0 & 0  & \frac{4}{36}\\ 
    \hline
    m & \frac{9}{36} & \frac{18}{36} & \frac{9}{36} & 1 
    \end{array}
\end{equation*}

La distribuzione di Z:
\begin{align*}
    &P(Z=0) = P(R=0,V=0) = \frac{1}{36} \\
    &P(Z=1) = P(R=0,V=1) + P(R=1,V=0) + P(R=1,V=1) = \frac{22}{36} \\
    &P(Z=2) = P(R=0,V=2) + P(R=2,V=0)  = \frac{13}{36} \\
\end{align*}
Il calcolo della varianza:
\begin{align*}
    &E[Z] = \frac{4}{3} \\
    &E[Z^2] = \frac{74}{36} \\
    &Var(Z) = \frac{5}{18}    
\end{align*}
    
\subsection*{Esercizio 5}

\subsection*{Esercizio 6}

\end{document}