\documentclass[12pt]{article}
\usepackage[top=2cm]{geometry}
\usepackage{amsmath}
\usepackage{amsthm}
\usepackage{amsfonts}
\usepackage{parskip}
\usepackage{graphicx}


\newtheorem{definition}{Definizione}[section]
\newtheorem{proposition}{Proposizione}[section]
\newtheorem{theorem}{Teorema}[section]
\newtheorem{nota}{Nota}[section]
\newtheorem{notaAMargine}{Nota a margine}[section]

\title{\textbf{Foglio 7}}
\author{Daniele Falanga}
\date{}

\begin{document}
\maketitle

\subsection*{Esercizio 1}

\begin{enumerate}
    \item La distribuzione del tempo di rottura del circuito in serie è dato dal componente che si rompe per primo:
    \[
    T_{ser} = min(T_1,\dots,T_k)    
    \]
    \item La distribuzione del tempo di rottura del circuito è dato dalla somma di tutti i componenti:
    \[
    T_{par} = \sum_{i}^{k}T_i    
    \]
\end{enumerate}

\subsection*{Esercizio 2}
Assumendo che su una popolazione di cento persone, in media \(\lambda = 2\), sono mancini, per calcolare la probabilità che almeno 3 siano mancini:

\begin{align*}
    &P(X \ge 3) = 1 - (P(X = 0) + P(X = 1) + P(X = 2)) = \\    
    &P(X \ge 3) = 1 - (\frac{2^0}{1}e^{-2} + \frac{2}{1}e^{-2} + \frac{2^2}{2}e^{-2}) = \\
    &P(X \ge 3) = 1-5e^{-2} = 0.32 = 32\%
\end{align*}  

\newpage
\subsection*{Esercizio 3}
Date le seguenti variabili aleatorie:
\begin{itemize}
    \item Y = y numero di mail spam 
    \item Z = z numero di mail non spam
    \item X = y+z numero di mail totale in arrivo  
\end{itemize}

La funzione di massa di probabilità congiunta:
\begin{align*}
    &P(Y=y, Z=z) = P(Y=y, X = y+z) \\
    &P(Y=y, X = y+z) = P(Y=y|X = y+z)\cdot P(X) \\    
\end{align*}

Modello le due probabilità in questo modo:
\begin{itemize}
    \item \(P(Y,y|X = y+z) = \binom{y+z}{y}\cdot p^y \cdot (1-p)^{y-z} \) : numero di modi possibili di scegliere y mail dalle y+z in arrivo per le rispettive probabilità
    \item \(P(X = y+z) = \frac{\lambda^{y+z}}{(y+z)!} e^{\lambda}\)
\end{itemize}

Sostituendo:
\begin{align*}
    &P(Y=y, X = y+z) = P(Y=y|X = y+z)\cdot P(X) = \\
    &\binom{y+z}{y}\cdot p^y \cdot (1-p)^{y-z} \cdot \frac{\lambda^{y+z}}{(y+z)!} e^{\lambda} = \\
    &\frac{(y+z)!}{y!z!}p^y(1-p)^z \cdot \frac{\lambda^y \lambda^z}{(y+z)!} e^{\lambda p (1-p)} = \\
    &\frac{(\lambda p)^y}{y!}e^{-\lambda p} \cdot \frac{(\lambda (1-p))^z}{z!}e^{-\lambda (1-p)} \\        
\end{align*}

Le distribuzioni marginali:
\begin{align*}
    &P(N_1 = y) = \sum_{z=0}^{\infty} P(P(Y=y, X = y+z)) = \\
    &\frac{(\lambda p)^y}{y!}e^{-\lambda p} \sum_{z=0}^{\infty}\frac{(\lambda (1-p))^z}{z!}e^{-\lambda (1-p)} = \\
    &\frac{(\lambda p)^y}{y!}e^{-\lambda p} \quad \text{La sommatoria è funzione di massa di una Poisson, quindi uguale ad 1}
\end{align*}
Analogamente:
\begin{align*}
    &P(N_2 = z) = \sum_{y=0}^{\infty} P(P(Y=y, X = y+z)) = \\
    &\frac{(\lambda (1-p))^z}{z!}e^{-\lambda (1-p)} \sum_{y=0}^{\infty} = \frac{(\lambda p)^y}{y!}e^{-\lambda p}\\
    &\frac{(\lambda (1-p))^z}{z!}e^{-\lambda (1-p)} 
\end{align*}

Esendo le marginali, il prodotto delle congiunte, le variabili sono indipendenti
\subsection*{Esercizio 4}

Scrivendo le variabili in questo modo:
\begin{itemize}
    \item X = numero medio lanci effettuati 
    \item Y = numero di teste ottenute
    \item Z = numero di croci ottenute 
\end{itemize}

Modellando come l'esercizio 3 si risolve in modo analogo

\subsection*{Esercizio 5}
Il processo in questione può essere descritto da una variabile geometrica, dove non teniamo conto della la probabilità 
che il primo successo (o evento in generale) richieda l'esecuzione di k prove indipendenti, bensi che il primo fallimento richieda l'esecuzione
di k prove indipendenti. Nel processo in questione però si tratta del lancio della moneta, quindi la probabilità di successo e di fallimento,
sono analoghe, quindi la distribuzione rimane la medesima:
\begin{enumerate}
    \item La distribuzione:
    \[
    P(X = k) = p(1-p)^k    
    \]
    \item Il valore atteso:
    \[
    E[X] = \frac{1}{p} = 2    
    \]
    \item La varianza:
    \[
    Var(X) = \frac{q}{p^2} = 6     
    \]
\end{enumerate}

\subsection*{Esercizio 6}

\end{document}