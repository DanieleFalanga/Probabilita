\documentclass[12pt]{article}
\usepackage[top=2cm]{geometry}
\usepackage{amsmath}
\usepackage{amsthm}
\usepackage{amsfonts}
\usepackage{parskip}
\usepackage{graphicx}


\newtheorem{definition}{Definizione}[section]
\newtheorem{proposition}{Proposizione}[section]
\newtheorem{theorem}{Teorema}[section]
\newtheorem{nota}{Nota}[section]
\newtheorem{notaAMargine}{Nota a margine}[section]

\title{\textbf{Foglio 7}}
\author{Daniele Falanga}
\date{}

\begin{document}
\maketitle

\subsection*{Esercizio 1}

\begin{enumerate}
    \item La distribuzione del tempo di rottura del circuito in serie è dato dal componente che si rompe per primo:
    \[
    T_{ser} = min(T_1,\dots,T_k)    
    \]
    \item La distribuzione del tempo di rottura del circuito è dato dalla somma di tutti i componenti:
    \[
    T_{par} = \sum_{i}^{k}T_i    
    \]
\end{enumerate}

\subsection*{Esercizio 2}
Assumendo che su una popolazione di cento persone, in media \(\lambda = 2\), sono mancini, per calcolare la probabilità che almeno 3 siano mancini:

\begin{align*}
    &P(X \ge 3) = 1 - (P(X = 0) + P(X = 1) + P(X = 2)) = \\    
    &P(X \ge 3) = 1 - (\frac{2^0}{1}e^{-2} + \frac{2}{1}e^{-2} + \frac{2^2}{2}e^{-2}) = \\
    &P(X \ge 3) = 1-5e^{-2} = 0.32 = 32\%
\end{align*}  

\newpage
\subsection*{Esercizio 3}

Si utilizza la formula:
\begin{align*}
    &P(A|B) = \frac{P(A \cap B)}{P(B)} = \\
    &P(A \cap B) = P(A|B){P(B)}
\end{align*}

Sostituendo in questo modo:
\begin{align*}
    \begin{cases}
        P(A \cap B) = P(Y=y, Z=z) \\
        P(A|B) = P(Y=y|X = x+z) \\
        P(B) = P(X = x+z)
    \end{cases}        
\end{align*}
Si ottiene:
\begin{align*}
    &P(Y=y, Z=z) = P(Y=y|X = x+z)\cdot P(X = x+z) = \\
    &\frac{X^{y+z}}{(y+z)!}e^{-x}\cdot \binom{y+z}{y}p^y(1-p)^z =\\
    &e^{-x}\frac{X{y+z}}{(y+z)!}\cdot\frac{(y+z)!}{y!z!}p^y(1-p)^z =\\
    &e^{-Xp-(1-p)}\frac{X^y}{y!}p^y\frac{X^z}{z!}(1-p)^z = \\
    &\underbrace{e^{-Xp}\frac{(Xp)^y}{y!}}_{Y \approx Poiss(Xp)}\cdot \underbrace{e^{X(1-p)}\frac{(X(1-p))^z}{z!}}_{Y \approx Poiss(X(1-p))}
\end{align*}

L'intersezione di due eventi indipendenti è la loro moltiplicazione e quindi, i due eventi sono indipendenti.

\subsection*{Esercizio 4}

Scrivendo le variabili in questo modo:
\begin{itemize}
    \item X = numero medio lanci effettuati 
    \item Y = numero di teste ottenute
    \item Z = numero di croci ottenute 
\end{itemize}

Modellando come l'esercizio 3 si risolve in modo analogo

\subsection*{Esercizio 5}
Il processo in questione può essere descritto da una variabile geometrica, dove non teniamo conto della la probabilità 
che il primo successo (o evento in generale) richieda l'esecuzione di k prove indipendenti, bensi che il primo fallimento richieda l'esecuzione
di k prove indipendenti. Nel processo in questione però si tratta del lancio della moneta, quindi la probabilità di successo e di fallimento,
sono analoghe, quindi la distribuzione rimane la medesima:
\begin{enumerate}
    \item La distribuzione:
    \[
    P(X = k) = p(1-p)^k    
    \]
    \item Il valore atteso:
    \[
    E[X] = \frac{1}{p} = 2    
    \]
    \item La varianza:
    \[
    Var(X) = \frac{q}{p^2} = 6     
    \]
\end{enumerate}

\subsection*{Esercizio 6}

\end{document}