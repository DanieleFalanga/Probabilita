\documentclass[12pt]{article}
\usepackage[top=2cm]{geometry}
\usepackage{amsmath}
\usepackage{amsthm}
\usepackage{amsfonts}
\usepackage{parskip}
\usepackage{graphicx}


\newtheorem{definition}{Definizione}[section]
\newtheorem{proposition}{Proposizione}[section]
\newtheorem{theorem}{Teorema}[section]
\newtheorem{nota}{Nota}[section]
\newtheorem{notaAMargine}{Nota a margine}[section]

\title{\textbf{Foglio 5}}
\author{Daniele Falanga}
\date{}

\begin{document}
\maketitle

\subsection*{Esercizio 1}
Dati gli eventi:
\begin{itemize}
    \item A pernice colpita da Alice
    \item B pernice colpito da Bob
    \item E pernice colpito da entrambi
    \item F pernice colpito da una freccia
\end{itemize}
\begin{enumerate}
    \item Non dando il numero di tentativi posso utilizzare la proprietà delle variabili indipendenti:
    \[
    P(E) = P(A)\cdot P(B) = 1/3 \cdot 1/2 = \frac{1}{12}   
    \]
    \item Posso usare il complemento:
    \begin{align*}
        &P(F) = 1-[P(A^c) \cdot P(B^c)] = \\
        &P(F) = 1 - \left[ \frac{2}{3} \cdot \frac{3}{4}\right] = \\
        &P(F) = \frac{1}{2}
    \end{align*}
    \item Uso la formula di Bayes:
    \begin{align*}
        \begin{cases}
            &P(A|F) = \frac{P(A)\cdot P(F|A)}{P(F)}  = \frac{4}{9}\\
            & \\
            &P(F|A) = \frac{P(F) \cap P(A)}{P(F)} = \frac{2}{3}
        \end{cases}
    \end{align*}    
\end{enumerate}


\subsection*{Esercizio 2}



\subsection*{Esercizio 3}


\subsection*{Esercizio 4}
\begin{enumerate}
    \item La distribuzione di X:
    \[
    P(X) = \sum_{i=1}^6 \frac{1}{6}   
    \]
    \item Il valore atteso:
    \[
    E[X] = \sum_{i = 1}^6 i\frac{1}{6}   
    \]
    \item Data la seguente formula di Varianza:
    \begin{align*}
        &Var(X) = E[(X-E[X])^2] \\
        &Var(X) = 2.91
    \end{align*}
\end{enumerate}
\subsection*{Esercizio 5}

\begin{enumerate}
    \item La distribuzione di X:
    \begin{align*}
        &P(X = 1) = \frac{11}{36} \\
        &P(X = 2) = \frac{9}{36} \\
        &P(X = 3) = \frac{7}{36} \\
        &P(X = 4) = \frac{5}{36} \\
        &P(X = 5) = \frac{3}{36} \\
        &P(X = 6) = \frac{1}{36} \\
    \end{align*}
    \item Il valore atteso: 
    \[
    E[X] = 2.36    
    \]
\end{enumerate}

\subsection*{Esercizio 6}
\begin{enumerate}
    \item Ogni domanda è una prova ripetuta. Le prove sono 10, per passare il test con 18
    bisogna rispondere bene a 7 domande (7 successi) e rispondere male a 3(3 fallimenti), quindi, si 
    usa la variabile binomiale:
    \[
    P(X=18) = \binom{10}{7}\left(\frac{1}{4}\right)^7\left(\frac{3}{4}\right)^3 = 0.30%    
    \]
    \item Il valore atteso:
    \[
    E[X] = np = 10\cdot \left(\frac{1}{4}\right) = 0.25 \%    
    \]
    \item La varianza:
    \[
    Var[X] = np(1-p) = 0.25\% \cdot \frac{3}{4}    
    \]    
\end{enumerate}

\end{document}