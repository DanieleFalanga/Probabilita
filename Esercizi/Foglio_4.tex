\documentclass[12pt]{article}
\usepackage[top=2cm]{geometry}
\usepackage{amsmath}
\usepackage{amsthm}
\usepackage{amsfonts}
\usepackage{parskip}
\usepackage{graphicx}


\newtheorem{definition}{Definizione}[section]
\newtheorem{proposition}{Proposizione}[section]
\newtheorem{theorem}{Teorema}[section]
\newtheorem{nota}{Nota}[section]
\newtheorem{notaAMargine}{Nota a margine}[section]

\title{\textbf{Foglio 4}}
\author{Daniele Falanga}
\date{}

\begin{document}
\maketitle

\subsection*{Esercizio 1}
\begin{enumerate}
    \item Denominando con F l'evento che il prodotto sia difettoso, utilizzando il teorema della probabilità
    totale:    
    \begin{align*}
        &P(F) = P(A)\cdot P(F|A) + P(B)\cdot P(F|B) + P(C)\cdot P(F|C) = \\
        &P(F) = 0.40\cdot 0.02 + 0.10 \cdot 0.03 + 0.50 \cdot 0.04 = \\
        &P(F) = 0.031 = 3.1 \%
    \end{align*}
    \item Utilizzando il teorema di Bayes per tutte e 3 le macchine:
    \begin{align*}
        &P(A|F) = \frac{P(A)P(F|A)}{P(F)} = (0.40 \cdot 0.02) = 0.031 = 0.25\\
        &P(B|F) = \frac{P(B)P(F|B)}{P(F)} = (0.10 \cdot 0.03) = 0.031 = 0.09\\
        &P(C|F) = \frac{P(C)P(F|C)}{P(F)} = (0.50 \cdot 0.04) = 0.031 = 0.64\\
    \end{align*} 
\end{enumerate}

\newpage
\subsection*{Esercizio 2}

Il testo fornisce i seguenti parametri:
\begin{itemize}
    \item \(T_1\) = trasmesso 1
    \item \(T_0\) = trasmesso 0
    \item \(R_1\) = ricevuto 1
    \item \(R_0\) = ricevuto 0
    \item \(P(T_1)\) = 0.55
    \item \(P(T_0)\) = 0.45
    \item \(P(R_1|T_1)\) = 0.91
    \item \(P(R_0|T_0)\) = 0.94
\end{itemize}
\begin{enumerate}
    \item Teorema della probabilità totale:
    \begin{align*}
        &P(R_1) = P(T_1)\cdot P(R_1|T_1)+P(T_0)P(R_1^c|T_0) = \\
        &(0.55 \cdot 0.91) + (0.45 \cdot 0.09) = 0.54 \\
    \end{align*}
    \item uguale:
    \begin{align*}
        &P(R_0) = P(T_0)\cdot P(R_0|T_0)+P(T_1)P(R_0^c|T_1) = \\
        &(0.45 \cdot 0.94) + (0.55 \cdot 0.06) = 0.45 \\
    \end{align*}
    \item Formula di Bayes:
    \begin{align*}
        P(T_1|R_1) = \frac{P(T_1) \cdot P(R_1|T_1)}{P(R_1)} = 0.93 \\
    \end{align*}
    \item Formula di Bayes:
    \begin{align*}
        P(T_0|R_0) = \frac{P(T_0) \cdot P(R_0|T_0)}{P(R_0)} = 0.92 \\
    \end{align*}
    \item La probabilità dell'errore di trasmissione:
    \begin{align*}
        &P(E) = P(T_1)\cdot P(R_0|T_1)+P(T_0)\cdot P(R_1|T_0) = \\
        &(0.55 \cdot 0.06) + (0.45 \cdot 0.09) = 
    \end{align*}
\end{enumerate}
\subsection*{Esercizio 3}
\begin{enumerate}
    \item Utilizzo il modello di variabile aleatoria geometrica:
    \[
        P(X=5) = \left(\frac{18}{37}\right)\left(1-\frac{18}{37}\right)^{4}
    \]
    \item Variabile binomiale:
    \begin{align*}
        &P(Y \ge 2) = 1 - P(Y < 2) \rightarrow \\
        &P(Y<2) = p_0+p_1 = \binom{10}{0}\left(\frac{18}{37}\right)^0\left(\frac{19}{37}\right)^{10} + \binom{10}{1}\left(\frac{18}{37}\right)^1\left(\frac{19}{37}\right)^9\\
    \end{align*}
    \item Ipotizzando di vincere 2 euro a partita:
    \[
    P(Z = 6) = \binom{10}{6}\left(\frac{18}{37}\right)^6\left(\frac{19}{37}\right)^4    
    \]
\end{enumerate}

\subsection*{Esercizio 4}
\begin{enumerate}
    \item La formula in questione rappresenta \(n\) estrazioni da un'urna con reinserimento,
    contenente \(N\) palline nere e \(B\) bianche. Al primo membro le estrazioni sono divise:
    \begin{align*}
        &\binom{B}{k} \rightarrow \text{numero di palline bianche con k-estrazioni} \\
        &\binom{N}{n-k} \rightarrow \text{numero di palline nere su n-k estrazioni}
    \end{align*}
    Al secondo membro le estrazioni sono unite
    \item Divido in due distribuzioni:
    La prima che calcola le teste nei primi n lanci e la seconda che calcola le teste nei secondi
    n lanci. 
    \begin{align*}
        &P(X_T = i) = \binom{n}{i} (\frac{1}{2})^i(\frac{1}{2})^{n-i}
        &P(X_C = i) = \binom{n}{i} (\frac{1}{2})^i(\frac{1}{2})^{n-i}
    \end{align*}
          
\end{enumerate}

\subsection*{Esercizio 5}
Denotando con \(p_1 = 1/2,p_2 = 0,p_3 = 1\) le probabilità che esca testa testa al lancio della moneta:
\begin{enumerate}
    \item Avendo 6 casi totale e 3 teste:
    \[
    P(T) = \frac{3}{6} = \frac{1}{2}    
    \]
    \item \(B\)- ha reso T; \(A\)-Faccia opposta è C; \(C_i\) estrazione moneta i-esima;
    \[
    P(A|B) = \frac{P(A\cup B)}{P(B)} = \frac{P(A \cup B | C_1)p(C_1) P(A \cup B | C_3)p(C_3) }{P(B)} = \frac{1/3}{1/2}    
    \]
\end{enumerate}
\subsection*{Esercizio 6}

Per semplificare la notazione, denotiamo con A e B i due sottoinsiemi scelti a caso da un insieme di cardinalità n. Ogni sottoinsieme può essere rappresentato come un vettore binario di lunghezza n, dove 1 rappresenta la presenza di un elemento e 0 rappresenta l'assenza.

Ad esempio, se abbiamo un insieme S con tre elementi {1, 2, 3}, allora il sottoinsieme {1, 3} può essere rappresentato come il vettore binario 101, mentre il sottoinsieme {2, 3} può essere rappresentato come il vettore binario 011.

Ora, per calcolare la probabilità che il primo sottoinsieme sia incluso nel secondo, dobbiamo considerare tutte le combinazioni dei vettori binari di lunghezza n e calcolare la probabilità che il primo sia incluso nel secondo.

Per ogni vettore binario di lunghezza n, ci sono \(2^n\) possibilità. Tuttavia, solo una di queste rappresenta il caso in cui il primo sottoinsieme è incluso completamente nel secondo. Ad esempio, se il primo sottoinsieme è rappresentato dal vettore binario 101, allora il secondo sottoinsieme dovrebbe essere rappresentato da un vettore binario 111 (tutti gli 1 nel secondo vettore corrispondono agli 1 nel primo vettore).

Quindi, la probabilità che il primo sottoinsieme scelto sia incluso completamente nel secondo è 1 su \(2^n\).

Pertanto, la probabilità richiesta è:

P(primo sottoinsieme incluso nel secondo) = \(1 / 2^n\)

\end{document}