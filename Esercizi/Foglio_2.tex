\documentclass[12pt]{article}
\usepackage[top=2cm]{geometry}
\usepackage{amsmath}
\usepackage{amsfonts}
\usepackage{parskip}
\newtheorem{definizione}{Definizione}[section]
\newtheorem{teorema}{Teorema}[section]
\newtheorem{nota}{Nota}[section]
\newtheorem{notaAMargine}{Nota a margine}[section]


\title{\textbf{Foglio 2}}
\author{Daniele Falanga}
\date{}

\begin{document}
\maketitle

\subsubsection*{Esercizio 1}
{\bf{1)}} 
Assegno prima il presidente tra i 25 candidati, poi il segretario tra i
24 rimanenti, per fare questo tipo di assegnazione ci sono: 
\[
25*24 = 600     
\]

modi per assegnare il presidente e il segretario

{\bf{2)}}
Ci sono 2 eventi: 
\begin{enumerate}
    \item A: assegnare presidente/segretario  
    \item B: assegnare presidente/segretario una volta assegnato uno dei due
\end{enumerate}
La probabilità di A è \(\frac{1}{25}\) 
\newline
Mentre la probabilità di B è \(\frac{1}{25} + \frac{1}{24}\) 
La probabilità totale quindi: 
\[
    p = \frac{1}{25} + \frac{1}{24} \simeq 0.08
\]

\subsubsection*{Esercizio 2}
Si risolve tramite il principio di enumerazione generalizzato: 
\begin{enumerate}
    \item RISO: 4 elementi, 4! diverse permutazioni 
    \item PATATE: 6!
    \item COZZE: 5!
\end{enumerate}

\subsubsection*{Esercizio 3}



\end{document}