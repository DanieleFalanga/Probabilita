\documentclass[12pt]{article}
\usepackage[top=2cm]{geometry}
\usepackage{amsmath}
\usepackage{amsthm}
\usepackage{amsfonts}
\usepackage{parskip}
\usepackage{graphicx}


\newtheorem{definition}{Definizione}[section]
\newtheorem{proposition}{Proposizione}[section]
\newtheorem{theorem}{Teorema}[section]
\newtheorem{nota}{Nota}[section]
\newtheorem{notaAMargine}{Nota a margine}[section]

\title{\textbf{Foglio 6}}
\author{Daniele Falanga}
\date{}

\begin{document}
\maketitle

\subsection*{Esercizio 1}
Dimostrazione a pagina 170
\subsection*{Esercizio 2}

Calcolo le probabilità singole, tramite il principio di enumerazione. Denotando con \(X_i\), 
il probabilità che esca un transistor rotto al lancio k-esimo:
\begin{align*}
    &P(X=1) = \frac{3}{10} \\
    &P(X=2) = \frac{7}{10} \cdot \frac{3}{9} \\
    &P(X=3) = \frac{7}{10} \cdot \frac{3}{8} \\
    &\vdots \\
\end{align*}
e cosi via fino ad \(P(X = 7)\) dove la probabilità è la stessa fino a 10 visto che stiamo pescando
sicuramente un transistor rotto. \newline
Calcolata la distribuzione di probabilità, si usa la formula del valore atteso: 
\[
E[X] = \sum_{1}^{10}i \cdot P(X = i)    
\]

\newpage
\subsection*{Esercizio 3}
{\bf{Punto 1}}: \newline
Siano X e Y due variabili aleatorie. Se X è una variabile aleatoria certa, 
ovvero \(X = c\) per un qualche \(c \in R\), allora la sua funzione di distribuzione di probabilità è:
\[
P(X = c) = 1  
\]
La funzione di distribuzione congiunta di X e Y è:
\[
P(X = c, Y = y) = P(X = c) P(Y = y | X = c)    
\]
Poiché X è certa, la probabilità che X assuma il valore c è 1. Inoltre, la probabilità che Y assuma 
il valore y dato che X ha assunto il valore c è la stessa che Y assuma il valore y indipendentemente dal valore di X. Pertanto, abbiamo:
\[
P(X = c, Y = y) = 1 \cdot P(Y = y)    
\]
Questo significa che la distribuzione di probabilità di Y è la stessa sia che X 
sia certa sia che X assuma un valore diverso da c. Pertanto, X e Y sono indipendenti.

{\bf{Punto 2}}: \newline
Essendo binarie si ha:
\begin{align*}
    X = 
    \begin{cases}
        1 \rightarrow E[X = p(x=1 = \frac{1}{2}) \\
        0
    \end{cases}        
    Y = 
    \begin{cases}
        1 \rightarrow E[X = p(x=1 = \frac{1}{2}) \\
        0
    \end{cases}        
\end{align*}

Una condizione di indipendenza di variabili aleatorie è la seguente:
\[
E[XY] = E[X]\cdot E[Y]    
\]
Questa condizione è possibile sfruttarla all'interno della dimostrazione:
\begin{align*}
    \begin{cases}
        &Cov(X,Y) = E[XY] - E[X]\cdot E[Y]  = 0\\
    \end{cases}
\end{align*}

\subsection*{Esercizio 4}
Utilizzo una distribuzione geometrica:
\begin{enumerate}
    \item La probabilità richiesta:
    \[
        P(T = 3, X = 5) = \frac{1}{6}\left( \frac{4}{6}\right)^2 = \frac{2}{27}  
    \]
    \item La distribuzione: 
    \begin{align*}
        &P(T=k) = \frac{2}{6} \left(\frac{4}{6}\right)^{k-1} \\
    \end{align*}
    \item La distribuzione di X:
    \[
        P(X=i) = \frac{1}{6}
    \]
    \item Se gli eventi sono indipendenti, la congiunta deve essere uguale al prodotto delle marginali:
    \begin{align*}
        P(T=3, X=5) = P(T = 3) \cdot P(X = 5) = \left(\frac{4}{6}\right)^2 \frac{2}{6} \cdot \frac{1}{6} = \frac{2}{81} \neq \frac{2}{27} \quad \text{punto 1} 
    \end{align*} 
\end{enumerate}

\subsection*{Esercizio 5}
Utilizzo 6 variabili aleatorie, \(X_i\) con \(i = 1,2,3\dots, 6\) per rappresentare il lancio i-esimo:
\begin{align*}
    &P(X = 1) = \frac{6}{6} \quad \text{Ho tutte le facce disponibili} \\
    &P(X = 2) = \frac{5}{6} \quad \text{Ne ho una di meno}\\
    &P(X = 3) = \frac{4}{6} \quad \text{e cosi via}\\
    &P(X = 4) = \frac{3}{6} \\
    &P(X = 5) = \frac{2}{6}\\
    &P(X = 6) = \frac{1}{6}\\
\end{align*}

Essendo ogni lancio indipendente uso la variabile geometrica 
\subsection*{Esercizio 6: \underline{Non so se è giusto}}

\begin{align}
    &E[X] = \sum_{j=2}^{n+n}(\sum_{i=1}^{n} f(i) \cdot \sum_{i=1}^{n} g(i)) \\
    &E[X^2] = \sum_{j=2}^{n+n}(\sum_{i=1}^{n} f(i)^2 \cdot \sum_{i=1}^{n} g(i)^2) \\    
    &E[X]^2 = (\sum_{j=2}^{n+n}(\sum_{i=1}^{n} f(i) \cdot \sum_{i=1}^{n} g(i)))^2 \\
    &Var(X) = \sum_{j=2}^{n+n}(\sum_{i=1}^{n} f(i)^2 \cdot \sum_{i=1}^{n} g(i)^2) - (\sum_{j=2}^{n+n}(\sum_{i=1}^{n} f(i) \cdot \sum_{i=1}^{n} g(i)))^2 
\end{align}

\end{document}