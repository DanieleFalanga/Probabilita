\documentclass[12pt]{article}
\usepackage[top=2cm]{geometry}
\usepackage{amsmath}
\usepackage{amsthm}
\usepackage{amsfonts}
\usepackage{parskip}
\usepackage{graphicx}


\newtheorem{definition}{Definizione}[section]
\newtheorem{proposition}{Proposizione}[section]
\newtheorem{theorem}{Teorema}[section]
\newtheorem{nota}{Nota}[section]
\newtheorem{notaAMargine}{Nota a margine}[section]

\title{\textbf{Foglio 6}}
\author{Daniele Falanga}
\date{}

\begin{document}
\maketitle

\subsection*{Esercizio 1}

\subsection*{Esercizio 2}

Dati \(N = 3\) transistor rotti e \(B = 7\) e transistori buoni.
Utilizzando il valore atteso della distribuzione ipergeometrica:
\[
E[X] = n \cdot \frac{3}{10} = 1 \cdot \frac{3}{10}    
\] 

\subsection*{Esercizio 3}

\subsection*{Esercizio 4}
Utilizzo una distribuzione geometrica:
\begin{enumerate}
    \item La probabilità richiesta:
    \[
    P(T = 3, X = 5) = \frac{1}{6}\left( \frac{5}{6}\right)^2 = 0.11   
    \]
    \item La distribuzione: 
    \begin{align*}
        &P(T=1, X=5) = \frac{1}{6} \left(\frac{5}{6}\right)^0 = 0.16 \\
        &P(T=2, X=5) = \frac{1}{6} \left(\frac{5}{6}\right)^1 = 0.13 \\
        &P(T=3, X=5) = \frac{1}{6} \left(\frac{5}{6}\right)^2 = 0.14 \\
    \end{align*}
    \item La distribuzione di X:
    \[
        P(T = 3, X=1,2,3,4) = 0.11
    \]
    \item Dal punto 2 del problema si nota come la probabilità che si verifichi un successo dipende dal numero di lanci effettuati.
    Quindi le due variabili sono dipendenti. 
\end{enumerate}
\subsection*{Esercizio 5}
Utilizzo 6 variabili aleatorie, \(X_i\) con \(i = 1,2,3\dots, 6\) per rappresentare il lancio i-esimo:
\begin{align*}
    &P(X = 1) = \frac{6}{6} \quad \text{Ho tutte le facce disponibili} \\
    &P(X = 2) = \frac{5}{6} \quad \text{Ne ho una di meno}\\
    &P(X = 3) = \frac{4}{6} quad \text{e cosi via}\\
    &P(X = 4) = \frac{3}{6} \\
    &P(X = 5) = \frac{2}{6}\\
    &P(X = 6) = \frac{1}{6}\\
\end{align*}

Essendo ogni lancio indipendente uso la variabile geometrica 
\subsection*{Esercizio 6}


\end{document}