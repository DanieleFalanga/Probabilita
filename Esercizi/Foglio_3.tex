\documentclass[12pt]{article}
\usepackage[top=2cm]{geometry}
\usepackage{amsmath}
\usepackage{amsthm}
\usepackage{amsfonts}
\usepackage{parskip}
\usepackage{graphicx}


\newtheorem{definition}{Definizione}[section]
\newtheorem{proposition}{Proposizione}[section]
\newtheorem{theorem}{Teorema}[section]
\newtheorem{nota}{Nota}[section]
\newtheorem{notaAMargine}{Nota a margine}[section]

\title{\textbf{Foglio 3}}
\author{Daniele Falanga}
\date{}

\begin{document}
\maketitle

\subsection*{Esercizio 1}
\begin{enumerate}
    \item se A e B sono disgiunti, vuol dire che la sua intersezione è uguale a 0
    utilizzando la formula:    
    \begin{align*}
        &P(A \cup B) = P(A) + P(B) - P(A \cap B) = \\
        &0.5 = 0.3 + P(B) = \\
        &P(B) = 0.2
    \end{align*}
    \item se A e B sono indipendenti, vuol dire che la loro intersezione vale \(P(A \cap B) = P(A)*P(B)\):
    \begin{align*}
        &P(A \cup B) = P(A) + P(B) - P(A \cap B) =  \\
        &P(A \cup B) = P(A) + P(B) - [P(A)P(B)] = \\ 
        &P(A \cup B) = P(B)[1-P(A)] + P(A) = \\
        &0.2 = P(B)[0.7] = \\
        &P(B) = 0.28
    \end{align*}        
    \item Se A è un sottoinsieme di B, la sua probabilità è inclusa in quella di B, e la loro unione
    restituisce quella di B:
    \begin{align*}
        &P(A \cup B) = P(B) = \\
        &P(B) = 0.5 
    \end{align*}
\end{enumerate}

\subsection*{Esercizio 2}

Se A,B e C sono indipendenti, allora ciascuno di essi risulta indipendente dagli altri 2:
\[
P[A \cap (B \cup C)] = P(A)P(B \cup C) \quad \text{Pag 87}    
\]
Inoltre, per la proposizione 3.8.2:
\begin{align*}
    &P(A^c \cap F) = P(A^c)P(F) \\
    &P(F) = P(B \cup C)    
\end{align*}

\newpage
In questo modo abbiamo dimostrato che \(A^c\) risulta indipendente dagli B e da C.
\newline
Per il \textbf{punto 2} dobbiamo dimostrare che \(B^c\) sia indipendente da \(A^c\) e da \(C\), ma siccome
abbiamo dimostrato prima che \(A^c\) e \(C\) sono indipendenti, al dimostrazione risulta analoga a prima, dimostrando quindi,
che \(B^c\) risulta indipendente dagli altri 2. 
Per il \textbf{punto 3}, dobbiamo dimostrare che \(C^c\) risulti indipendente dagli altri due, siccome per
\(B^c\) e \(A^c\) abbiamo verificato essere indipendenti, la dimostrazione risulta analoga.
\subsection*{Esercizio 3}
\subsubsection*{Punto a}

La vittoria dei tre cavalli sono 3 variabili aleatorie congiunte:
\begin{itemize}
    \item X = vittorie del cavallo a
    \item Y = vittore di b
    \item Z = vittorie di c
\end{itemize}

Se volessi calcolare la probabilità di vittoria dello stesso cavallo:
\begin{align*}
    P(X=3,Y=3,Z=3) = (0.3)^3 + (0.5)^3 + (0.2)^3
\end{align*}

\subsubsection*{Punto b}
\[
    P(X=1,Y=1,Z=1) = \binom{3}{1,1,1}(0.3) (0.5) (0.2) =    
\]
\subsection*{Esercizio 4}
\subsubsection*{Punto a}
Utilizzo il modello di variabile aleatoria binomiale
\[
P(X=i) = \binom{n}{i} (p)^i (1-p)^{n-i}    
\]
X = Colpisco il bersaglio una volta
\[
P(X \ge 1) = 1 - P(X=0) = 1 - \binom{3}{0} (\frac{1}{3})^0 (\frac{2}{3})^3   
\]  

\subsubsection*{Punto b}
Non sono sicuro però penso si faccia cosi
\newline
Calcolo dal complementare
\begin{align*}
    P(X<1) = \left(\frac{2}{3}\right)^n < 0.10 \\
    n\log(\frac{2}{3}) < 0,10 \\
    n > 6
\end{align*}

\subsection*{Esercizio 5}

\subsection*{Esercizio 6}


\end{document}