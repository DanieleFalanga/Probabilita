\documentclass[12pt]{article}
\usepackage[top=2cm]{geometry}
\usepackage{amsmath}
\usepackage{amsthm}
\usepackage{amsfonts}
\usepackage{parskip}
\usepackage{graphicx}


\newtheorem{definition}{Definizione}[section]
\newtheorem{proposition}{Proposizione}[section]
\newtheorem{theorem}{Teorema}[section]
\newtheorem{nota}{Nota}[section]
\newtheorem{notaAMargine}{Nota a margine}[section]

\title{\textbf{Foglio 3}}
\author{Daniele Falanga}
\date{}

\begin{document}
\maketitle

\subsection*{Esercizio 1}


\subsection*{Esercizio 2}


\subsection*{Esercizio 3}


\subsection*{Esercizio 4}
\subsubsection*{Punto a}
Utilizzo il modello di variabile aleatoria binomiale
\[
P(X=i) = \binom{n}{i} (p)^i (1-p)^{n-i}    
\]
X = Colpisco il bersaglio una volta
\[
P(X=1) = \binom{3}{1} \left(\frac{1}{3}\right)^1 \left(\frac{2}{3}\right)^2 = \frac{4}{9}   
\]  

\subsubsection*{Punto b}
Non sono sicuro però penso si faccia cosi
\newline
Calcolo dal complementare
\begin{align*}
    P(X<1) = \left(\frac{2}{3}\right)^n < 0.10 \\
    n\log(\frac{2}{3}) < 0,10 \\
    n > 6
\end{align*}


\end{document}